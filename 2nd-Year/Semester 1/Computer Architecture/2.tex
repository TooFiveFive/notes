\documentclass[12pt]{article}
\usepackage{hyperref}
\begin{document}
{\centering
\section*{Computer Architecture}
\section*{Lecture 2}
\indent\today
}

\subsection*{Instruction Set}
\begin{itemize}
    \item The instrcutions fed to a CPU
    \item Used for arithmetic / logical, memory + port transfers and flow control.
\end{itemize}
\subsection*{Instruction set design}
Approaches: \textbf{RISC} (\textit{Reduced Instruction Set Computer}) and \textbf{CISC} (\textit{Complex Instruction Set Computer})
\subsection*{RISC}
\begin{itemize}
    \item Simple instructions that only take about 1 clock cycle
\end{itemize}
\subsection*{CISC}
\begin{itemize}
    \item Complex multi clock instructions.
    \item x86 etc
\end{itemize}

\subsection*{MIPS}
\begin{itemize}
    \item Has 32x32-bit registers.
    \item Used for Values of resultsm arguments, temps, global pointer, stack pointer, program counter, etc.
\end{itemize}

\subsection*{Integer arithmetic}
\begin{itemize}
    \item \textit{MIPS} can process both integer and floating point numbers
    \item It has both 32 and 64 nit architectures.
    \item When a 32-bit integer is \textbf{signed}, \textbf{the most significant bit} (\textit{bit 31}) is used to denote negativity.
    \item 2's compliment is used to invert the most significant bit and invert the entire number (\textit{see notes})
\end{itemize}

\subsection*{MIPS Instructions}
\begin{itemize}
    \item Can only have \textbf{3 operands}.
    \item Eg: \textit{add \$0 \$S0 \$S2}
\end{itemize}

\subsection*{MIPS R-format instructions}
\textit{R-formated} just means that the instruction is encoded using a format which means:
\newline
The op takes 6 bits, the registers then take 5 bits, the shift takes 5 bits, and the funct takes 6 bits
\newline
\newline
\href{https://en.wikibooks.org/wiki/MIPS_Assembly/Instruction_Formats}{See wiki for more info.}

\subsection*{Memory Operands}
\begin{itemize}
    \item MIPS memory is \textbf{byte-addressed}. You can load single bytes from memory.
\end{itemize}

\end{document}
